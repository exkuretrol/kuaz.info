\documentclass[11pt,a4paper,sans]{moderncv}

% 設定 style 與顏色
\moderncvcolor{blue}  % 仍需指定,但會被覆寫
\moderncvstyle{classic}

% 邊界
\usepackage[hmargin=1cm,vmargin=1cm]{geometry}
\usepackage{setspace}
\setstretch{1.2} % adjust global spacing


% 字體
\usepackage{xeCJK}
\usepackage{fontspec}
\usepackage{unicode-math}
\defaultfontfeatures{Ligatures=TeX}
\setCJKmainfont[
  UprightFont = Noto Sans TC Regular,
  BoldFont    = Noto Sans TC Bold
]{Noto Sans TC}
\xeCJKsetup{CJKglue=\hskip 0pt plus .08\baselineskip}


% 標題改黑色
\colorlet{firstnamecolor}{black}


% 個人資料
\name{陳家瑋}{}
\phone[mobile]{+886~922~765~255}
\email{kuaz@kuaz.info}
\homepage{kuaz.info}
\social[github]{exkuretrol}

\begin{document}
\makecvtitle

% 專業簡介
\section{專業簡介}
\cvitem{}{具備 Linux 系統管理、伺服器維運與容器化經驗,曾參與 AI 計算平台與網站伺服器的建置與維護。熟悉 Docker、GitOps、CI/CD 與監控工具,並具備全端開發背景,能跨足系統穩定性與服務開發,致力於以自動化提升平台可靠性。}

% 研究興趣
\section{研究興趣}
\cvitem{}{系統維運、自動化部署、伺服器管理、容器化、雲端平台}

% 教育背景
\section{學歷}
\cventry{2020--2024}{應用統計與資料科學學系}{銘傳大學金融科技學院}{}{}{雙主修:資訊工程學系}

% 工作經驗
\section{工作經驗}
\cventry{2024--至今}{Django 全端工程師}{好物有限公司}{}{}{
\begin{itemize}
  \item 開發與維護 Django Web 系統,涵蓋前端與後端功能。
  \item 維護電商平台同步系統,並開發藥局總部內部管理系統。
  \item 導入 CI/CD Pipeline,自動化部署與測試,將伺服器更新時間由 30 分鐘縮短至 5 分鐘。
  \item 整合 GitHub Actions 與 Claude AI,自動審查 Pull Request 程式碼品質,縮短 Code Review 時間。
  \item 基礎設施平台建置與維護:
    \begin{itemize}
      \item 部署 Proxmox 虛擬化環境,管理 LXC 容器資源。
      \item 部署 Gitea 作為自建程式碼託管平台。
      \item 佈署 Grafana、Grafana Alloy、Prometheus 建立監控與可觀測性平台。
      \item 實作自動化流程,將服務部署至 staging 環境的 LXC。
    \end{itemize}
  \item 支援 Linux 伺服器環境管理,確保應用服務穩定。
\end{itemize}
}

\cventry{2020--2023}{教學助理}{銘傳大學}{}{}{協助課程:資料科學概論、程式語言、探索性資料分析與視覺化、統計資料庫、網站伺服器概論、應用模擬方法、網際網路資料庫。支援學生實作與伺服器環境操作。}

\cventry{2022}{資料工程實習生}{明台產物保險}{}{}{使用 SQL 清理與處理客戶保單資料,提升數據分析效率。}

\cventry{2021}{專案實習}{台灣郵政大數據準備室}{}{}{撰寫 SQL 語法,協助建立客戶清單。}

% 平台建置與維運經驗
\section{平台建置與維運經驗}
\cventry{2025--至今}{AI 平台維護}{}{}{}{管理銘傳大學應統系 aiWorks AI 訓練工作站,負責 Linux 系統維護、平台工作區容器建置,支援師生深度學習模型訓練。}

\cventry{2022--至今}{網站伺服器建置}{}{}{}{建置並維護銘傳大學桃園行政處與統計系 WordPress 網站。}

\cventry{2024--至今}{容器化與自動化}{}{}{}{使用 Docker、GitOps 與 Ansible 進行服務部署,搭配 Grafana/Prometheus 監控系統效能,提升維護效率與可靠性。}

% 技能
\section{技能}
\cvitem{系統}{Linux (Ubuntu/Debian)、CI/CD (GitHub Actions)、Ansible(初階)}
\cvitem{虛擬化}{Proxmox、LXC、Docker}
\cvitem{版本控管}{Gitea、GitOps、GitHub Actions}
\cvitem{監控}{Grafana、Grafana Alloy、Prometheus}
\cvitem{程式}{Python、Shell Script、R、C、C++、\LaTeX}
\cvitem{WebD}{Django、HTML、CSS、JavaScript、Tailwind CSS}

% 語言
\section{語言能力}
\cvitemwithcomment{英文}{TOEIC 675 (2023)}{具備閱讀英文技術文件與 API 文件的能力}

% References
\section{References}
\begin{cvcolumns}
  \cvcolumn{李御璽}{%
    專任教授 \\
    銘傳大學資訊工程學系 \\
    \href{mailto:leeys@mail.mcu.edu.tw}{leeys@mail.mcu.edu.tw}%
  }
  \cvcolumn{王薔惠}{%
    助理教授 \\
    銘傳大學金融科技應用學系 \\
    \href{mailto:mandyw@mail.mcu.edu.tw}{mandyw@mail.mcu.edu.tw}%
  }
\end{cvcolumns}


\end{document}
